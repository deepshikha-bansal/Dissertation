\lstset { %
	language=C,
	backgroundcolor=\color{lightgray}, % set backgroundcolor
	%basicstyle=\footnotesize,% basic font setting
	basicstyle=\ttfamily\scriptsize,
	%\basicstyle=\ttfamily\scriptsize,
	keywordstyle=\color{blue}\ttfamily,
	stringstyle=\color{red}\ttfamily,
	commentstyle=\color{darkgray}\ttfamily,
	breaklines=true	
}
\lstset{framesep=-10pt, xleftmargin=-10pt}
\begin{lstlisting}[caption={FHEW: FFT Setup Function},label={listing:3.8.3}]
//FFT Setup function
void FFTsetup(){

//Sets up data to compute single precision float FFT

  in_float = (float*) fftwf_malloc(sizeof(float) * 2*N);
  out_float = (fftwf_complex*) fftwf_malloc(sizeof(fftwf_complex) * (N + 2));
  plan_fft_forw_float = fftwf_plan_dft_r2c_1d(2*N, in_float, out_float,  FFTW_PATIENT);
  plan_fft_back_float = fftwf_plan_dft_c2r_1d(2*N, out_float, in_float,  FFTW_PATIENT);

//Sets up data to compute single precision float FFT

  in_double = (double*) fftw_malloc(sizeof(double) * 2*N);
  out_double = (fftw_complex*) fftw_malloc(sizeof(fftw_complex) * (N + 2));
  plan_fft_forw_double = fftw_plan_dft_r2c_1d(2*N, in_double, out_double,  FFTW_PATIENT);
  plan_fft_back_double = fftw_plan_dft_c2r_1d(2*N, out_double, in_double,  FFTW_PATIENT);
}

\end{lstlisting}



