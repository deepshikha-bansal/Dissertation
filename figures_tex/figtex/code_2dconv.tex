
\lstset { %
	language=C,
	backgroundcolor=\color{lightgray}, % set backgroundcolor
	%basicstyle=\footnotesize,% basic font setting
	basicstyle=\ttfamily\scriptsize,
	%\basicstyle=\ttfamily\scriptsize,
	keywordstyle=\color{blue}\ttfamily,
	stringstyle=\color{red}\ttfamily,
	commentstyle=\color{darkgray}\ttfamily,
	breaklines=true	
}
\lstset{framesep=-10pt, xleftmargin=-10pt}
\begin{lstlisting}[caption={Two Dimensional Convolution Operation Code Snippet},label={listing:61}]

out.at<float>(r,c) = 0;

for(int i = 0; i < ker.rows; i++) {

  for(int j = 0; j < ker.cols; j++) {   
    
    //Define a fixed point variable "a"
    //to store the fixed point value of kernel
    Fi::Fixed<word_length, frac_length, Fi::SIGNED> a(ker.at<float>(i, j));
    
    //Define a fixed point variable "b" 
    //to store the fixed point value of Input
    Fi::Fixed<word_length, frac_length, Fi::SIGNED> b(input.at<float>(r+i, c+j));
    
    //Define a fixed point variable "temp1" 
    //to store the result of fixed point multiplication
    Fi::Fixed<word_length, frac_length, Fi::SIGNED> temp1(a*b);
    
    //the result is converted back to float and added to the output.
	out.at<float>(r,c) += temp1.toFloat();
	
	}
}
\end{lstlisting}
